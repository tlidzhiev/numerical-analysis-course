\ProvidesFile{lecture-02.tex}[Лекция 2]

\newpage

\section{Прямые методы решения систем линейных уравнений}

\subsection*{Постановка задачи}

Рассмотрим систему уравнений:

\begin{equation}\label{SLE}
    \begin{cases}
        A_{11}x_1 + \ldots A_{1n}x_n = y_1\\
        \ldots\\
        A_{n1}x_1 + \ldots + A_{nn}x_n = y_n
    \end{cases}
\end{equation}

Система уравнений \eqref{SLE} называется линейной системой уравнений (СЛУ). $A_{ij}$ --- матрица коэффициентов, $y_i$ --- правая часть и $x_i$ --- неизвестные

Про СЛУ естественно говорить на языке линейных пространств и операторов: пусть задан линеный оператор $A: X \longrightarrow Y$, вектор $y \in Y$ и требуется найти его прообраз $x \in X: Ax = y$

\subsection*{Предположения}

В общем случае при решении СЛУ возможны несколько случаев:

\begin{itemize}
    \item Решение существует и единственно $\ker A = 0$ и $y \in \Im A$

    \item Решение существует, но не единственно $\ker A \neq 0$ и $y \in \Im A$

    \item Решения не существует $y \notin \Im A$
\end{itemize}

В данном курсе мы будем рассматривать только частный случай: $\ker A = 0$ и $\Im A = Y$ или $\mathrm{coker} \, A = 0$

Иными словами: будем считать, что $X$ и $Y$ --- линейные пространства одинаковой размерности $n$, матрица СЛУ $A$ является несингулярной (квадратной и невырожденной)

В такой постановке задача $Ax=y$ определена корректно и решение существует и единственно для любой правой части

Далее в курсе будут разбираться различные алгоритмы поиска решения в зависимости от свойств матрицы $A$

\subsection{Нижние и верхние треугольные матрицы}

\begin{definition}
Матрица $L$ называется \textit{нижней треугольной} матрицей, если $L_{ij} = 0$, если $j > i$
\end{definition}

\begin{definition}
Матрица $U$ называется \textit{верхней треугольной} матрицей, если $U_{ij} = 0$, если $j < i$
\end{definition}

Нижние и верхние треугольные матрицы обладают следующим важным свойством

\begin{claim}
Пусть $A$ и $B$ нижние (верхние) треугольные матрицы, то и матрица $C = AB$ является нижней (верхней) треугольной матрицей
\end{claim}

\begin{proof}
    Для доказательства вычислим значение матричного элемента $C_{ij}$ при $j > i$ для нижних треугольных матриц:

    \[
    C_{ij} = \sum\limits_{k = 1}^n A_{ik} \cdot B_{kj} = \sum\limits_{k = 1}^i A_{ik} \cdot B_{kj} + \sum\limits_{k = i+1}^n A_{ik} \cdot B_{kj} = \sum\limits_{k = 1}^i A_{ik} \cdot 0 + \sum\limits_{k = i+1}^n 0 \cdot B_{kj} = 0
    \]
\end{proof}

\subsubsection*{Решение системы с нижней треугольной матрицей}

Рассмотрим СЛУ с нижней треугольной матрицей $L$:

\begin{equation*}\label{lowerTrigSLE}
    \begin{cases}
        L_{11}x_1 &= y_1\\
        L_{21}x_1 + L_{22}x_2 &=y_2\\
        \ldots\\
        L_{n1}x_1 + \ldots + L_{nn}x_n &= y_n
    \end{cases}
\end{equation*}

Тогда решение можно найти, исключая неизвестные:

\begin{equation*}
\begin{aligned}
    &x_1 = y_1 / L_{11}\\
    &x_2 = (y_2 - L_{21}x_1) / L_{22}\\
    &\ldots\\
    &x_i = \left(y_i - \sum\limits_{j = 1}^{i - 1} L_{ij}x_j\right) / L_{ii}\\
    &\ldots
\end{aligned}
\end{equation*}

\subsubsection*{Решение системы с верхней треугольной матрицей}

Рассмотрим СЛУ с верхней треугольной матрицей $U$:

\begin{equation*}\label{upperTrigSLE}
    \begin{cases}
        U_{11}x_1 + \ldots + U_{1n}x_n &= y_1\\
        \ldots\\
        U_{(n-1)(n-1)}x_{n-1} + \ldots + U_{(n-1)n}x_{n} &= y_{n-1}\\
        U_{nn}x_1 &= y_n
    \end{cases}
\end{equation*}


Тогда решение можно найти, исключая неизвестные:

\begin{equation*}
\begin{aligned}
    &x_n = y_n / U_{nn}\\
    &x_{n-1} = (y_{n-1} - U_{(n-1)n}x_{n-1}) / U_{(n-1)(n-1)}\\
    &\ldots\\
    &x_i = \left(y_i - \sum\limits_{j = i+1}^n U_{ij}x_j\right) / U_{ii}\\
    &\ldots
\end{aligned}
\end{equation*}

\subsection{Решение системы уравнений с помощью LU разложения}

Пусть есть СЛУ $Ax = y$

Идея алгоритма состоит в следующем:

\begin{itemize}
    \item Представить матрицу $A$ в виде произведения: $A = LU$, где $L$ --- нижняя треугольная матрица, $U$ --- верхняя треугольная
    \item Решить систему $Lz = y$
    \item Решиить систему $Ux = z$
\end{itemize}

\begin{proof}
    Пусть $x$ --- решение исходной системы. Тогда

    \[
    Ax = LUx = L(Ux) = Lz = y
    \]
\end{proof}

\paragraph{Замечание}

\begin{itemize}
    \item В общем случае LU разложение не определено однозначно: если $D$ --- невырожденная диагональная матрица, то можно построить другое LU разложение по уже имеющемуся:

    \[
    A = LU = LDD^{-1}U = (LD)(D^{-1}U) = L'U'
    \]

    Данную неопределенность можно решить, зафиксировав, что $L_{ii} = 1$ или $U_{ii} = 1$
\end{itemize}

\subsubsection*{Алгоритм построения LU разложения}

Рассмотрим случай, когда диагональ верхней треугольной матрицы $U_{ii} = 1$. Будем вычислять элементы матриц $L$ и $U$ построчно:

\begin{itemize}
    \item Первая строка: $L_{11}U_{11} = A_{11}$ и $U_{11} = 1$, поэтому $L_{11} = A_{11} / U_{11}$ (перемножили 1-ую строку и 1-ый столбец). Далее вычислим $U_{1i}$: (перемножаем 1-ую строку и $i$-ый столбец)

    \[
    L_{11}U_{1i} = A_{1i} \iff U_{1i} = A_{1i} / L_{11}
    \]

    \item Вторая строка: $L_{21}U_{11} = A_{21}$ и $U_{11} = 1$, поэтому $L_{21} = A_{21} / U_{11}$ (перемножили 1-ую строку и 2-ый столбец). Далее вычислим $L_{22}$: (перемножим 2-ую строку и 2-ый столбец)

    \[
    L_{21}U_{12} + L_{22}U_{22} = A_{22} \iff L_{22} = (A_{22} - L_{21}U_{12}) / U_{22}
    \]

    Теперь вычислим $U_{2i}$: (перемножим 2-ую строку и $i$-ый столбец)

    \[
    L_{21}U_{12} + L_{22}U_{2i} = A_{2i} \iff U_{2i} = (A_{2i} - L_{21}U_{12}) / L_{22}
    \]

    И так далее

    \item В итоге получаем, что вычислить $i$-ую строку матрицы $L$, можно следующим образом: $(j \leqslant i)$

    \[
    L_{ij} = \left(A_{ij} - \sum\limits_{k = 1}^{j - 1} L_{ik}U_{kj}\right) / U_{jj}
    \]

    А $i$-ая строка матрицы $U$ вычисляется так: $(j > i)$

    \[
    U_{ij} = \left(A_{ij} - \sum\limits_{k = 1}^{i-1}L_{ik}U_{kj}\right) / L_{ii}
    \]
\end{itemize}

\paragraph{Замечание}

Если мы задаем диагональные элементы нижней треугольной матрицы $L$, то задача сводится к уже решенной:

\[
A^{\top} = L'U' \iff A = (U')^{\top} (L')^{\top} = LU
\]

Данный алгоритм позволяет найти LU разложение для матрицы $A$ единственным образом, если зафиксировать диагональные элементы матрицы $U$ или $L$. Однако если на $m$-ом шаге окажется, что $L_{mm} = 0$, то алгоритм позволяет найти только лишь LU разложение главного минора порядка $m$ матрицы $A$:

\[[A]_m = [L]_m [U]_m,\]

где $[L]_m$ и $[U]_m$ --- нижняя и верхняя треугольные матрицы, вычисленные за $m$ шагов

В общем случае, данный алгоритм не гарантирует сходимости к LU разложению для матрицы $A$, однако существует класс матриц $A$, для которых алгоритм корректно находит LU разложение

\subsubsection*{Класс матриц}

Необходимо проверить, есть ли нули на диагонали матрицы $L$, тогда вышеописанный алгоритм будет работать корректно

\begin{claim}
    Если $\forall \, m \in \{1, \ldots n\}: \det [A]_m \neq 0$, то $L_{mm} \neq 0$
\end{claim}

\begin{proof}
    Докажем от противного. Зафиксируем $m \in \{1, \ldots, n\}$ и $\det [A]_m \neq 0$. Пусть $L_{ii} \neq 0$ при $i < m$ и $L_{mm} = 0$. Тогда верно, что

    \[
    [A]_m = [L]_m [U]_m \implies \det [A]_m = \det [L]_m \cdot \det [U]_m = \prod\limits_{i = 1}^m L_{ii} \cdot \prod\limits_{i = 1}^m U_{ii} = 0
    \]

    Получили противоречие, что определитель главного минора $[A]_m$ не равен $0$
\end{proof}

В общем случае данное утверждение сложно проверить. Если матрица является симметричной положительно определенной (SPD), то тогда оно выполнено по \href{https://en.wikipedia.org/wiki/Sylvester%27s_criterion}{критерию Сильвестра}

Помимо SPD матриц, часто встречаются матрицы со следующим особым свойством

\begin{definition}
    Матрица $A$ называется \textit{матрицей с диагональным преобладанием}, если $\forall \, i \in \{1, \ldots, n\}$

    \[
    |A_{ii}| - \sum\limits_{j \neq i} |A_{ij}| > 0
    \]
\end{definition}

Видно, что главный минор такой матрицы также является матрицей с диагональным преобладанием. Докажем следующее утверждение

\begin{claim}
    Матрица с диагональным преобладанием является несингулярной
\end{claim}

\begin{proof}
    Докажем от противного. Пусть матрица вырожденная. Тогда $\exists \, x \in \mathbb{R}^n: Ax = 0$. Найдем максимальный по модулю элемент в векторе $x: |x_i| = \max\limits_{j} |x_j|$ и рассмотрим $i$-ую строку $Ax$:

    \[
    0 = \left|\sum\limits_{j} A_{ij}x_j\right| = |x_i| \cdot \left|\sum\limits_{j} A_{ij} \frac{x_j}{|x_i|}\right| =
    %
    |x_i| \cdot \left|A_{ii} + \sum\limits_{j \neq i} A_{ij} \cdot \frac{x_j}{|x_i|}\right| \geqslant |x_i| \cdot \left|A_{ii} - \sum\limits_{j \neq i} |A_{ij}| \cdot \frac{|x_j|}{|x_i|}\right| > 0
    \]

    Предпоследнее неравенство верно по обратному неравенства треугольника. Последнее неравенство верно, так как матрица $A$ является матрицей с диагональным преобладанием, а отношение $|x_j| / |x_i| < 1$ при $j \neq i$.

    Получили противоречие
\end{proof}


\subsection{Решение системы уравнений с помощью разложения Холецкого}

Пусть есть СЛУ $Ax = y$ и матрица $A$ является SPD матрицей. Тогда существует специальный вид (причем единственный) LU разложения --- разложение Холецкого:

\[A = LL^{\top},\]

где $L$ --- нижняя треугольная матрица с положительными элементами на диагонали

\begin{claim}
    Существование разложения Холецкого
\end{claim}

\begin{proof}
    Пусть задано какое-то LU разложение: $A = LU$ (существование его доказывалось ранее). Тогда верно следующее:

    \[
    LU = A = A^{\top} = U^{\top} L^{\top}
    \]

    Домножим слева равенство на $L^{-1}$:

    \[
    U = L^{-1} U^{\top} L^{\top}
    \]

    Теперь домножим справа равенство на $(L^{\top})^{-1}$:

    \[
    U (L^{\top})^{-1} = L^{-1} U^{\top} = D
    \]

    Получим, что слева у нас верхняя треугольная матрица, а справа нижняя треугольная матрица, поэтому и справа, и слева диагональная матрица $D$

    Рассмотрим исходное LU разложение:

    \[
    A = LU = L \cdot (D L^{\top}) = L D^{1/2} D^{1/2} L^{\top} = (L D^{1/2}) (L D^{1/2})^{\top} = L' L'^{\top}
    \]

    Причем диагональные элементы $D$ могут быть только положительными, так как $A$ является SPD матрицей и $L$ --- матрица перехода от одного базиса к другому

\end{proof}

\begin{claim}
    Единственность разложения Холецкого
\end{claim}

\begin{proof}
    Единственность доказывается вместе с построением алгоритма вычисления, аналогичному для LU разложения

    \begin{itemize}
        \item $L_{11}L_{11} = A_{11} \iff L_{11} = \sqrt{A_{11}}$

        \item $i$-ая строка при $j < i$:

        \[
        \sum\limits_{k = 1}^{j - 1} L_{ik} L_{kj} + L_{ij}L_{jj} = A_{ij} \iff L_{ij} = \left(A_{ij} - \sum\limits_{k = 1}^{j - 1} L_{ik} L_{kj}\right) / L_{jj}
        \]

        \item Диагональные элементы $L_{ii}$:

        \[
        L_{ii} = \sqrt{A_{ii} - \sum\limits_{k = 1}^{j - 1} L_{ik} L_{kj}}
        \]
    \end{itemize}
\end{proof}

\subsection{Решение систем уравнений с помощью алгоритма Гаусса}

Пусть есть СЛУ $Ax = y$

Будем решать ее алгоритмом Гаусса: сначала применять прямой алгоритм
Гаусса, потом обратный. Данный алгоритм является одним из способов построения LU разложения

Во время прямого алгоритма Гаусса мы будем приводить матрицу $A$ к ступенчатому виду, выполняя операции над строками:

\begin{itemize}
    \item 1 тип: $i \mapsto i \cdot \lambda, \lambda \neq 0$
    \item 2 тип: $i \leftrightarrow j$
    \item 3 тип: $i \mapsto i + j \cdot \lambda$
\end{itemize}


Во время обратного хода мы теми же действиями будем приводить матрицу $A$ к улучшенному ступенчатому виду, чтобы главные коэффициенты были равны $1$

Каждой операции над строками однозначно сопоставляется умножение слева на матрицу

Сложность алгоритма Гаусса составляет $O(n^3)$, где $n$ --- разномерность матрицы $A$. Поэтому данные алгоритмы не используются для решения СЛУ больших размерностей
